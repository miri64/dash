\documentclass{beamer}

\usepackage[english]{babel}
\usepackage[T1]{fontenc}
\usepackage{graphicx}
\usepackage{ifthen}
\usepackage{listings}
\usepackage{verbatim}
\usepackage{fancyhdr}
\usepackage{tikz}
\usepackage{dsfont}
\usetikzlibrary{spy,matrix}
\usepackage[style=mla, backend=bibtex]{biblatex}
\bibliography{storm}

\usepackage{xunicode}
\usepackage{xltxtra}
\usepackage{setspace}
\defaultfontfeatures{Mapping=tex-text}
\setmonofont[Mapping={}, Scale=MatchLowercase]{DejaVu Sans Mono}
\setsansfont[Scale=MatchLowercase]{Linux Biolinum O}
\setmainfont[]{Linux Libertine O}

\newbox\mytempbox
\newdimen\mytempdimen
\newcommand\includegraphicscopyright[3][]{%
  \leavevmode\vbox{\vskip3pt\raggedright\setbox\mytempbox=\hbox{%
  \includegraphics[#1]{#2}}%
    \mytempdimen=\wd\mytempbox\box\mytempbox\par\vskip1pt%
    \fontsize{3}{3.5}\selectfont{\color{black!25}{\vbox{\hsize=\mytempdimen#3}}}\vskip3pt%
}}

\newcommand\prelim[1]{\small }

\newcommand\strColor[1]{\color{beamer@blendedblue}{#1}}

\newcommand\sect[1]{\begin{center}\huge\strColor{#1}\end{center}}

\setbeamertemplate{navigation symbols}{}

\pagestyle{fancy}
\rhead{\vspace{.3em}\includegraphics[width=10em]{assets/FULogo_RGB.eps}}
\cfoot{}
\renewcommand{\headrulewidth}{0pt}

\lstset{language=C}
\lstset{basicstyle=\footnotesize\ttfamily}
\lstset{breaklines=true}
\lstset{keywordstyle=\color{purple}}

\usepackage{geometry}
\usepackage{tikz}
\usetikzlibrary{calc,arrows}

\newdimen\XCoord
\newdimen\YCoord

\newcommand*{\ExtractCoordinate}[1]{\path (#1); \pgfgetlastxy{\XCoord}{\YCoord};}%

\title{POSIX compliant shell}
\author{Robert Fels, Martine Lenders, Jakob Pfender}
\institute{FU Berlin\\Institut für Informatik\\Softwareprojekt
Telematik}
\date{\today}

\begin{document}

\frame{\titlepage}

\frame{
\frametitle{Project topic}
\begin{itemize}
  \item Port a POSIX compliant shell (e.g. dash) to RIOT
  \begin{itemize}
    \item in order to provide an execution environment for shell scripts
  \end{itemize}
  \item Prerequisite: RIOT's POSIX compliance needs to be improved
\end{itemize}
}

\frame{
  \frametitle{dash}
  \begin{itemize}
    \item Debian Almquist Shell (derivate of BSD Almquist Shell)
    \item fast shell script execution
    \item Used by Debian (+derivatives) as default /bin/sh
    \item lightweight
  \end{itemize}
}

\frame{
    \frametitle{Goals}
    \begin{itemize}
      \item dash should build on RIOT
      \item dash should \textbf{work} on RIOT
      \item Side effect: RIOT should be more POSIX compliant as a result
    \end{itemize}
}

\frame{
    \frametitle{Work schedule -- Three phases}
    \begin{itemize}
      \item \textbf{Phase 1:} Compilation
        \begin{itemize}
          \item \textbf{Subphase 1:} Identify missing headers
          \item \textbf{Subphase 2:} Implement missing headers
        \end{itemize}
      \item \textbf{Phase 2:} Functionality
        \begin{itemize}
          \item \textbf{Subphase 3:} Identify missing functionality
          \item \textbf{Subphase 4:} Implement missing functionality
          \item \textbf{Subphase 5:} Test
        \end{itemize}
    \end{itemize}
     
}

\frame{
  \frametitle{Missing functionality (identified so far)}
  \begin{itemize}
    \item Pattern matching
    \item File system
    \item Signaling
    \item DEFINEs (errno, stddef etc.)
  \end{itemize}
}

\begin{frame}[fragile]
  \frametitle{Timeline}
    \pgfmathsetmacro{\mintime}{0}
    \pgfmathsetmacro{\maxtime}{11}
    \newcommand{\timeunit}{Weeks}
    \pgfmathtruncatemacro{\timeintervals}{11}
    \pgfmathsetmacro{\scaleitemseparation}{4}
    \pgfmathsetmacro{\timenodewidth}{2cm}
    \newcounter{itemnumber}
    \setcounter{itemnumber}{0}
    \newcommand{\lastnode}{n-0}

    \newcommand{\timeentry}[2]{% time, description
    \stepcounter{itemnumber}
    \node[below right,minimum width=\timenodewidth] (n-\theitemnumber) at (\lastnode.south west) {#2};
    \node[right] at (n-\theitemnumber.east) {};

    \edef\lastnode{n-\theitemnumber}

    \expandafter\edef\csname nodetime\theitemnumber \endcsname{#1}
    }

    \newcommand{\drawtimeline}{%
        \draw[very thick,-latex] (0,0) -- ($(\lastnode.south west)-(\scaleitemseparation,0)+(0,-1)$);
        \ExtractCoordinate{n-\theitemnumber.south}
        \pgfmathsetmacro{\yposition}{\YCoord/28.452755}
        \foreach \x in {1,...,\theitemnumber}
        {   \pgfmathsetmacro{\timeposition}{\yposition/(\maxtime-\mintime)*\csname nodetime\x \endcsname}
            \draw (0,\timeposition) -- (0.5,\timeposition) -- ($(n-\x.west)-(0.5,0)$) -- (n-\x.west);
        }
        \foreach \x in {0,...,\timeintervals}
        {   \pgfmathsetmacro{\labelposition}{\yposition/(\maxtime-\mintime)*\x}
            \node[left] (label-\x) at (-0.2,\labelposition) {\x\ \timeunit};
            \draw (label-\x.east) -- ++ (0.2,0);
        }   
    }

    \scalebox{0.7}{
      \begin{tikzpicture}
        \node[inner sep=0] (n-0) at (\scaleitemseparation,0) {};
        \timeentry{2.5}{May 11th: Successful compilation}
        \timeentry{3.5}{May 18th: Identify missing functionality}
        \timeentry{8}{June 29th: Implement functionality}
        \timeentry{10}{July 8th: Finish testing}
        \timeentry{11}{July 15th: Finish project}
        \drawtimeline
      \end{tikzpicture}
    }
    
\end{frame}

\end{document}
